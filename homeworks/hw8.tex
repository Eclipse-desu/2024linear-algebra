\title{第九周作业}
\author{洪艺中}
\maketitle
\section{第一部分}
\newcommand{\lvec}[1]{\overrightarrow{#1}}
\subsection*{ 习题 5.1 1 奇数 }
\begin{solution}
\begin{enumerate}
    \item[(1)] 
    \[
        3x^2 + 3y^2 + 8xy - 7 = 0, x \geqslant -y;
    \]
    \item[(3)] 
    \[
        \mlr(\theta) = ((a - b) \cos \theta + b \cos (\frac{b - a}{b}\theta), (a - b) \sin \theta + b \sin (\frac{b - a}{b}\theta));
    \]
    \item[(5)]
    \[
        \mlr(\theta) = ((a + b) \cos \theta - b \cos (\frac{a + b}{b}\theta), (a +  b) \sin \theta - b \sin (\frac{a + b}{b}\theta)).
    \]
\end{enumerate}

注: (3) 和 (5) 的答案是认为小圆初始位置在大圆右边, 球心逆时针移动, 定点 $P$ 初始位置为大圆和小圆的交点.
\end{solution}

\subsection*{ 题目 2 偶数 }
\begin{solution}
\begin{enumerate}
    \item[(2)] 设两点分别为 $(-c, 0, 0)$ 和 $(c, 0, 0)$. 到定点距离之和(差)为 $2a$. 如果 $c < a(c > a)$, 那么这个图形是椭圆(双曲线)绕 $Ox$ 轴旋转得到的旋转体. 设 $r^2 := y^2 + z^2$, 那么 $x$ 和 $r$ 满足椭圆(双曲线)方程, 即
    \[
    \frac{x^2}{a^2} + \frac{r^2}{a^2 - c^2} = 1,
    \]
    所以方程是
    \[
        \frac{x^2}{a^2} + \frac{y^2 + z^2}{a^2 - c^2} = 1.
    \]

    如果 $c = a$, 那么是和为定值时, 动点在两定点的中点处上, 即 $(0, 0, 0)$. 差为定值时, 动点在 $Ox$ 轴上两点以外的部分, 即 $x = t, t \leqslant -c \text{或}  t \geqslant c$. 其他情况下轨迹不存在.
    \item[(4)] $(x - 4)^2 + y^2 + (z - 1)^2 = 1$.
\end{enumerate}
\end{solution}

\subsection*{ 题目 3 }
\begin{solution}
    本题的答案按照 $xOy$, $yOz$, $zOx$ 交线的顺序给出.
\begin{enumerate}
    \item[(2)] $x^2 + 2y^2 = 6$ 椭圆; $2y^2 - 4z^2 = 6$ 双曲线; $x^2 - 4z^2 = 6$ 双曲线;
    \item[(4)] $(0, 0)$ 一个点; $y^2 = z$, $z \geqslant 0$ 抛物线; $x^2 = z$, $z \geqslant 0$ 抛物线; 
    \item[(5)] $(0, 0)$ 一个点; $y^2 = 4z^2$ 两条相交的直线; $x^2 = 4z^2$ 两条相交的直线.
\end{enumerate}
\end{solution}

\subsection*{ 题目 5 }
\begin{solution}
关于 $xOy$, $yOz$, $zOx$ 平面的射影柱面分别是
\[
(x - \frac{1}{2})^2 + y^2 = \frac{9}{4},
\]
\[
y^2 + (z - \frac{5}{2})^2 = \frac{9}{4},
\]
\[
z = x + 2, x \in [-1, 2].
\]
\end{solution}

\subsection*{ 题目 6 }
\begin{solution}
\[
\begin{cases}
    y = 2x, \\
    9x^2 + z^2 = 1.
\end{cases}
\]
\end{solution}

\section{第二部分}
\subsection*{ 习题 5.2 题目 1 偶数 }
\begin{solution}
\begin{enumerate}
    \item[(2)]
    坐标式参数方程是 
    \[
        \begin{cases}
            x = \lambda; \\
            y = \lambda; \\
            z = \mu.
        \end{cases}
    \]
    一般方程是
    \[
        x - y = 0.
    \]
    \item[(4)] 
    坐标式参数方程是 
    \[
        \begin{cases}
            x = 3 + \lambda + \mu, \\
            y = -5 + 6\lambda - 6\mu, \\
            z = 1 + \lambda + 3\mu.
        \end{cases}
    \]
    一般方程是
    \[
        12x - y - 6z = 35.
    \]
\end{enumerate}
\end{solution}

\subsection*{ 题目 2 }
\begin{solution}
截距方程
\[
\frac{x}{3} + \frac{y}{2} + \frac{z}{6} = 1,
\]
坐标式参数方程
\[
\begin{cases}
    x = \lambda, \\
    y = \mu, \\
    z = 6 - 2\lambda - 3\mu.
\end{cases}
\]
\end{solution}

\subsection*{ 题目 3 }
\begin{solution}
设截距式方程是
\[
\frac{x}{a} + \frac{y}{b} + \frac{z}{c} = 1,
\]
因为
\[
    \frac{1}{a} + \frac{1}{b} + \frac{1}{c} = k,
\]
所以过定点 $(\frac{1}{k}, \frac{1}{k}, \frac{1}{k})$.
\end{solution}

\subsection*{ 题目 5 }
\begin{solution}
\[
2x - y + z = 4 \pm \sqrt{6}.
\]
\end{solution}

\subsection*{ 题目 9 }
\begin{solution}
设平面上到原点最近的点是 $P$, 那么 $OP$ 和平面垂直, 并且 $|OP| = p$. 因为平面的法向是 $(\frac{1}{a}, \frac{1}{b}, \frac{1}{c})$, 所以设 
\[
    \lvec{OP} = k(\frac{1}{a}, \frac{1}{b}, \frac{1}{c}),
\]
它在平面上, 所以
\[
    k(\frac{1}{a^2} + \frac{1}{b^2} + \frac{1}{c^2}) = 1,
\]
同时
\[
    |\lvec{OP}|^2 = k^2(\frac{1}{a^2} + \frac{1}{b^2} + \frac{1}{c^2}) = p^2,
\]
结合上述两个方程, 解出 $k = p^2$. 因此
\[
    p^2(\frac{1}{a^2} + \frac{1}{b^2} + \frac{1}{c^2}) = 1,
\]
即
\[
    \frac{1}{a^2} + \frac{1}{b^2} + \frac{1}{c^2} = \frac{1}{p^2}.
\]
\end{solution}

\subsection*{ 题目 10 (1) }
\begin{solution}
\[
-19x^2 + 14y^2 + 5z^2 + 44xy - 10xz - 20yz + 70x - 160y + 50z - 25 = 0.
\]
\end{solution}

\subsection*{ 题目 11 }
\begin{solution}
    根据题目中的说法, 我们认为 $P_0$ 在 $P_1P_2$ 中间
\[
\frac{|P_0P_1|}{|P_0P_2|} = \frac{|Ax_1 + By_1 + Cz_1 + D|}{|Ax_2 + By_2 + Cz_2 + D|} = -\frac{Ax_1 + By_1 + Cz_1 + D}{Ax_2 + By_2 + Cz_2 + D}.
\]
\end{solution}

\subsection*{ 习题 5.3 题目 1(2) }
\begin{solution}
    因为标准方程可以很容易地转化为参数方程和一般方程, 所以答案只给出标准方程.
\begin{enumerate}
    \item[(2)]
    \[
        \frac{x - 2}{0} = \frac{y + 2}{1} = \frac{z - 1}{0};
    \]
    \item[(4)]
    \[
        \frac{x - 1}{1} = \frac{y}{1} = \frac{x - 1}{-3}.
    \]
\end{enumerate}
\end{solution}

\subsection*{ 题目 2(1) }
\begin{solution}
\[
\frac{x + 6}{-7} = \frac{y}{2} = \frac{z - 6}{5}.
\]
\end{solution}

\subsection*{ 题目 3 奇数 }
\begin{solution}
\begin{enumerate}
    \item[(1)] $x + y - z - 2 = 0$;
    \item[(3)] 关于 $xOy$, $yOz$, $zOx$ 平面的射影平面分别是 $x - 11y + 4 = 0$, $7y + z - 3 = 0$, $7x + 11z - 5 = 0$.
\end{enumerate}
\end{solution}

\subsection*{ 题目 5 (2) }
\begin{solution}
$(-1, 2, 1)$.
\end{solution}

\subsection*{ 题目 7 }
\begin{solution}
联立这些平面方程, 设系数矩阵是 $A$, 增广矩阵是 $\bar{A}$, 则这些平面通过同一条直线的充要条件是方程组有解, 且解至少有一个自由变量. 即 $r(A) = r(\bar{A}) \leqslant 2$.
\end{solution}