\title{第六周作业}
\author{洪艺中}
\maketitle
\section{第一次作业}
\subsection*{72页 题目4 (1)}
\begin{problem*}
设 $\field$ 是一个数域, 矩阵 $\mat{A} \in \field^{m \times m}$, $\mat{B} \in \field^{m \times n}$, $\mat{C} \in \field^{n \times m}$, $\mat{D} \in \field^{n \times n}$. 如果 $\mat{A}$, $\mat{D}$ 可逆, 证明
\[
|\mat{A} + \mat{B}\mat{D}^{-1}\mat{C}||\mat{D}| = |\mat{A}||\mat{D} + \mat{C}\mat{A}^{-1}\mat{B}|.
\]
\end{problem*}
\begin{solution}
构造
\[
\mat{M} := 
\begin{pmatrix}
    \mat{A} & \mat{B} \\
    -\mat{C} & \mat{D}
\end{pmatrix}.
\]

由于 $\mat{A}$ 可逆, 借助分块矩阵的初等行变换, 用第一行的 $\mat{A}$ 消掉第二行的 $\mat{C}$ 得到
\[
\begin{pmatrix}
    \mat{E}_m &  \\
    \mat{C}\mat{A}^{-1} & \mat{E}_n
\end{pmatrix}
\begin{pmatrix}
    \mat{A} & \mat{B} \\
    -\mat{C} & \mat{D}
\end{pmatrix}
=
\begin{pmatrix}
    \mat{A} & \mat{B} \\
            & \mat{D} + \mat{C}\mat{A}^{-1}\mat{B}
\end{pmatrix},
\]
所以
\[
|\mat{M}| = |\mat{A}||\mat{D} + \mat{C}\mat{A}^{-1}\mat{B}|.
\]

由于 $\mat{D}$ 可逆, 借助分块矩阵的初等行变换, 用第二行的 $\mat{D}$ 消掉第一行的 $\mat{B}$ 得到
\[
\begin{pmatrix}
    \mat{E}_m & -\mat{B}\mat{D}^{-1} \\
     & \mat{E}_n
\end{pmatrix}
\begin{pmatrix}
    \mat{A} & \mat{B} \\
    -\mat{C} & \mat{D}
\end{pmatrix}
=
\begin{pmatrix}
    \mat{A} + \mat{B}\mat{D}^{-1}\mat{C} &  \\
    -\mat{C} & \mat{D} 
\end{pmatrix},
\]
所以
\[
|\mat{M}| = |\mat{A} + \mat{B}\mat{D}^{-1}\mat{C}||\mat{D}|.
\]

综上可知
\[
    |\mat{A}||\mat{D} + \mat{C}\mat{A}^{-1}\mat{B}| = |\mat{A} + \mat{B}\mat{D}^{-1}\mat{C}||\mat{D}|.
\]
\end{solution}

\newpage
\subsection*{题目 5}
\begin{problem*}
设 $\mat{A}$ 是 $n$ 阶可逆矩阵, $\mat{\alpha}$, $\mat{\beta}$ 是两个 $n$ 元列向量. 证明:
\[
\left| \mat{A} + \mat{\alpha}\mat{\beta}^{\top} \right| = |\mat{A}|(1 + \mat{\beta}^{\top}\mat{A}^{-1}\mat{\alpha}).
\]
\end{problem*}
\begin{solution}
利用上一个题的结论. 取题目 4 中的 $\mat{A}$、$\mat{B}$、$\mat{C}$、$\mat{D}$ 分别为 $\mat{A}$、$\mat{\alpha}$、$\mat{\beta}^{\top}$、$\mat{E}_1$. 所以
\[
    |\mat{A}||\mat{D} + \mat{C}\mat{A}^{-1}\mat{B}| = |\mat{A}||\mat{E}_1 + \mat{\beta}^{\top}\mat{A}^{-1}\mat{\alpha}|,
\]
\[
    |\mat{A} + \mat{B}\mat{D}^{-1}\mat{C}||\mat{D}| = |\mat{A} + \mat{\alpha}\mat{\beta}^{\top}|,
\]
所以
\[
    \left|\mat{A} + \mat{\alpha}\mat{\beta}^{\top}\right| = |\mat{A}||\mat{E}_1 + \mat{\beta}^{\top}\mat{A}^{-1}\mat{\alpha}|.
\]
右边项里, $\mat{E}_1 + \mat{\beta}^{\top}\mat{A}^{-1}\mat{\alpha}$ 是一个 $1 \times 1$ 的矩阵. 根据矩阵的加法、乘法以及数乘的法则, 可以构造 $1 \times 1$ 矩阵空间 $\field^{1 \times 1}$ 和数域 $\field$ 之间的一个双射 $\mathrm{i} \colon \field^{1 \times 1} \to \field$, 把矩阵 $(a)$ 映射到 $a$. 这个映射不仅是双射, 它还保持了两个空间上的加法、乘法运算以及所有的运算性质, 例如交换律、分配律等等. 这说明 $1 \times 1$ 矩阵空间可以视为与数域等同的, 即 $\mat{E}_1 + \mat{\beta}^{\top}\mat{A}^{-1}\mat{\alpha}$ 也可以看成一个数 $1 + |\mat{\beta}^{\top}\mat{A}^{-1}\mat{\alpha}|$. 有了这种观念, 我们就可以自由地混用 $1 \times 1$ 矩阵和数了. 因此可以把结论写成
\[
    \left| \mat{A} + \mat{\alpha}\mat{\beta}^{\top} \right| = |\mat{A}|(1 + \mat{\beta}^{\top}\mat{A}^{-1}\mat{\alpha}).
\]
\end{solution}

\newpage
\subsection*{76页 题目1}
\begin{enumerate}
    \item 2;
    \item 4;
    \item 5;
    \item \begin{flalign*} & \begin{cases}
        3 & k \not= 2 \\
        2 & k = 2
    \end{cases}. &
\end{flalign*}
\end{enumerate}

\subsection*{题目 2}
\begin{problem*}
设 $n$ 阶非奇异矩阵 $\mat{A}$ 中每行元素之和都等于常数 $c$, 证明 $c \not= 0$ 且 $\mat{A}^{-1}$ 中每行元素之和都等于 $c^{-1}$.
\end{problem*}
\begin{solution}
首先如果 $c = 0$, 那么矩阵的行列式为 $0$, 则矩阵不可逆. 因此 $c \not= 0$.

接着考虑 $\mat{A}^{-1}$ 中元素的和. $\mat{A}^{-1} = \mat{A}^{\ast} / |\mat{A}|$, 那么第 $i$ 行元素的和可以表示为
\[
\dfrac{1}{|\mat{A}|}\sum_{j = 1}^{n} \mat{A}_{ji}, \tag{$\ast$}
\]
其中 $\mat{A}_{ji}$ 表示 $\mat{A}$ 的代数余子式. 这个形式让我们想到利用 Cramer 法则.

设
\[
\mat{x} := 
\begin{pmatrix}
    x_1 \\
    x_2 \\
    \vdots \\
    x_n
\end{pmatrix}
, \qquad
\mat{b} :=
\begin{pmatrix}
    1 \\
    1 \\
    \vdots \\
    1
\end{pmatrix},
\]
并且设 $\mat{B}_i$ 表示用 $\mat{b}$ 替换 $\mat{A}$ 的第 $i$ 列得到的矩阵. 那么线性方程组 $\mat{A}\mat{x} = \mat{b}$ 的解可以表示为 
\[
x_i = \dfrac{|\mat{B}_i|}{|\mat{A}|}. \tag{$\ast\ast$}
\]
而 $\mat{A}$ 每行的所有元素之和为 $c$, 所以
\[
\mat{x} := 
\begin{pmatrix}
    c^{-1} \\
    c^{-1} \\
    \vdots \\
    c^{-1}
\end{pmatrix}
\]
是方程的一个解, 同时因为 $\mat{A}$ 可逆, 其也是唯一解. 因此根据 ($\ast\ast$) 式,对任何 $i$,
\[
    c^{-1} = \dfrac{|\mat{B}_i|}{|\mat{A}|},
\]
而
\[
    \dfrac{|\mat{B}_i|}{|\mat{A}|} = \dfrac{1}{|\mat{A}|}\sum_{j = 1}^{n}\mat{A}_{ji},
\]
观察上式和 ($\ast$) 式, 可见两者是一样的. 所以 $\mat{A}^{-1}$ 每行元素的和就是 $c^{-1}$.
\end{solution}

\newpage
\section{第二部分}
\subsection*{80页 题目1(3)}
首先可以很容易验证这样的结论:
\begin{proposition}
    对于二阶矩阵 $\begin{pmatrix} a & b \\ c & d \end{pmatrix}$, 其可逆的充要条件是 $ad - bc = 0$. 如果可逆, 那么其逆为
    \[
    \dfrac{1}{ad - bc}
    \begin{pmatrix}
        d & -b \\
        -c & a
    \end{pmatrix}.
    \]
    也就是经过以下三步得到
    \begin{enumerate}
        \item 交换主对角的两个元素 $a$, $d$ 的位置;
        \item 给副对角的两个元素 $b$, $c$ 加上负号;
        \item 乘上行列式 $ad - bc$ 的倒数.
    \end{enumerate}
\end{proposition}

回到题目. 有了刚才的工具, 可以瞬间算出
\[
\begin{pmatrix}
    1 & 4 \\ -1 & 2
\end{pmatrix}^{-1}
=\frac{1}{6}
\begin{pmatrix}
    2 & -4 \\ 1 & 1
\end{pmatrix}, \qquad
\begin{pmatrix}
    2 & 0 \\ -1 & 1
\end{pmatrix}^{-1}
=\frac{1}{2}
\begin{pmatrix}
    1 & 0 \\ 1 & 2
\end{pmatrix},
\]
所以
\[
\mat{X} = \frac{1}{6}
\begin{pmatrix}
    2 & -4 \\ 1 & 1
\end{pmatrix}
\begin{pmatrix}
    3 & 1 \\ 0 & -1
\end{pmatrix}
\frac{1}{2}
\begin{pmatrix}
    1 & 0 \\ 1 & 2
\end{pmatrix}
=
\begin{pmatrix}
    1 & 1 \\ \frac{1}{4} & 0
\end{pmatrix}.
\]

\subsection*{题目 2}
\[
    \mat{A}_1^{-1} = \left(\begin{matrix}
        \frac{1}{3} & \frac{1}{6} & \frac{2}{3} \\
        \frac{1}{3} & \frac{1}{6} & \frac{-1}{3} \\
        \frac{-1}{3} & \frac{1}{3} & \frac{1}{3}
    \end{matrix}\right), \qquad 
    \mat{A}_2^{-1} = \left(\begin{matrix}
        \frac{-1}{8} & \frac{1}{4} & \frac{-3}{8} \\
        \frac{1}{2} & 0 & \frac{-1}{2} \\
        \frac{-1}{8} & \frac{1}{4} & \frac{5}{8}
    \end{matrix}\right)
\]

\newpage
\subsection*{83页 题目2}
\begin{problem*}
设 $\mat{A} \in \field^{r \times r}$, $\mat{B} \in \field^{s \times s}$, $\mat{C} \in \field^{s \times r}$, 证明 $r\left(\begin{pmatrix} \mat{A} & \mat{O} \\ \mat{O} & \mat{B} \end{pmatrix}\right) \leqslant r\left(\begin{pmatrix} \mat{A} & \mat{O} \\ \mat{C} & \mat{B} \end{pmatrix}\right)$.
\end{problem*}
\begin{proof}
右边矩阵的秩是 $r(\mat{A}) + r(\mat{B})$. 为了证明这个不等式, 我们只需要在右边的矩阵中找到一个 $r(\mat{A}) + r(\mat{B})$ 阶的非零子式. 记 $s := r(\mat{A})$, $t = r(\mat{B})$. 设 
$\mat{A}\begin{pmatrix}
    i_1 & i_2 & \cdots & i_s \\
    j_1 & j_2 & \cdots & j_s
\end{pmatrix}$ 是 $\mat{A}$ 的一个 $s$ 阶非零子式, 
$\mat{B}\begin{pmatrix}
    k_1 & k_2 & \cdots & k_t \\
    l_1 & l_2 & \cdots & l_t
\end{pmatrix}$ 是 $\mat{B}$ 的一个 $t$ 阶非零子式. 那么取不等式右边矩阵(记为 $\mat{M}$)的 $r + s$ 阶子式
$\mat{M}\begin{pmatrix}
    i_1 & i_2 & \cdots & i_s & r + k_1 & r + k_2 & \cdots & r + k_t \\
    j_1 & j_2 & \cdots & j_s & r + l_1 & r + l_2 & \cdots & r + l_t
\end{pmatrix}$
这个子式可以表示为
\[
    \begin{pmatrix}
        \mat{A}_{r \times r} & \mat{O}_{r \times s} \\
        \mat{C}_{s \times r} & \mat{B}_{s \times s}
    \end{pmatrix},
\]
其中 $\mat{A}_{r \times r}$ 和 $\mat{B}_{s \times s}$ 就是从 $\mat{A}$ 和 $\mat{B}$ 中选取的那两个子式. 而 $\mat{C}_{s \times r}$ 中的元素都来自 $\mat{C}$. 根据分块矩阵的行列式计算规律, 这个子式的行列式为 $|\mat{A}_{r \times r}||\mat{B}_{s \times s}| \not= 0$. 所以这就说明了 $r(\mat{M}) \geqslant r + s$. 因此原不等式得证.
\end{proof}

\newpage
\subsection*{题目 3}
\begin{problem*}
证明
\[
r(\mat{A}) + r(\mat{B}) \geqslant r\left(\begin{matrix}\mat{A} \\ \mat{B} \end{matrix}\right) \geqslant \max\{r(\mat{A}, r(\mat{B}))\}.
\]
\end{problem*}
\begin{proof}
设 $\mat{A}$ 是 $m_1 \times n$ 的,  $\mat{B}$ 是 $m_2 \times n$ 的. 并记 $\mat{M} := \left(\begin{matrix}\mat{A} \\ \mat{B} \end{matrix}\right)$.

先证明右边的不等号. 因为 $\mat{A}$ 和 $\mat{B}$ 的子式都是 $\mat{M}$ 的子式, 所以 $r(\mat{M}) \geqslant r(\mat{A})$ 且 $r(\mat{M}) \geqslant r(\mat{B})$. 因此 $r(\mat{M}) \geqslant \max\{r(\mat{A}, r(\mat{B}))\}$.

再证明左边的不等号. 利用初等变换可以把 $\mat{M}$ 中的 $\mat{B}$ 部分变为标准形, 即若
\[
    \mat{P}\mat{B}\mat{Q} = \begin{pmatrix}
        \mat{E_s} & \mat{O} \\
        \mat{O} & \mat{O}
    \end{pmatrix},
\]
那么
\[
\begin{pmatrix}
    \mat{E}_{m_1} & \\
    & \mat{P}
\end{pmatrix}
\begin{pmatrix}
    \mat{A} \\
    \mat{B}
\end{pmatrix}
\mat{Q}
=
\begin{pmatrix}
    \multicolumn{2}{c}{\mat{A}\mat{Q}} \\
    \mat{E_s} & \mat{O} \\
    \mat{O}_{(m_2 - s) \times s} & \mat{O}_{(m_2 - s) \times (n - s)}
\end{pmatrix}
\]

因为添加一个零行, 不会出现更高阶的非零子式, 所以矩阵的秩不会变化. 因此上面这个矩阵的最后 $m_2 - s$ 行去掉也不会影响矩阵的秩. 因此 $\mat{M}$ 的秩相比 $\mat{A}$ 的秩最多增加 $s$. 所以其秩不大于 $r(\mat{A}) + s = r(\mat{A}) + r(\mat{B})$. 即
\[
r(\mat{A}) + r(\mat{B}) \geqslant r\left(\begin{matrix}\mat{A} \\ \mat{B} \end{matrix}\right).
\]

因此不等式得证.
\end{proof}

\newpage
\subsection*{86页 题目3}
题目略.
\begin{proof}
    记这个方程组为 $\mat{A}\mat{x} = \mat{b}$, 则题目中的条件等于是说
    \[\mat{B} = 
    \begin{pmatrix}
        \mat{A} & \mat{b} \\
        \mat{b}^\top & 0
    \end{pmatrix}
    \]
    的秩等于 $\mat{A}$ 的秩. 那么 $\begin{pmatrix} \mat{A} & \mat{b} \end{pmatrix}$ 作为 $\mat{B}$ 的子阵, 其秩不超过 $\mat{B}$ 的秩, 所以 $r\begin{pmatrix} \mat{A} & \mat{b} \end{pmatrix} \leqslant r(\mat{B}) = r(\mat{A})$; 而 $\mat{A}$ 又是 $\begin{pmatrix} \mat{A} & \mat{b} \end{pmatrix}$ 的子阵, 所以 $r\begin{pmatrix} \mat{A} & \mat{b} \end{pmatrix} \geqslant r(\mat{A})$. 综上 $r\begin{pmatrix} \mat{A} & \mat{b} \end{pmatrix} = r(\mat{A})$, 因此方程组有解.
\end{proof}

\subsection*{题目4}
题目略.
\begin{proof}
    考虑方程组
    \[
        \left\{
        \begin{aligned}
            a_{11} x_1 + a_{12} x_2 + \cdots + a_{1n} x_n + b_1 x_{n+1} = 0 \\
            a_{21} x_1 + a_{22} x_2 + \cdots + a_{2n} x_n + b_2 x_{n+1} = 0 \\
            \vdots \\
            a_{n1} x_1 + a_{n2} x_2 + \cdots + a_{nn} x_n + b_n x_{n+1} = 0 \\
            a_{n+1, 1} x_1 + a_{n+1, 2} x_2 + \cdots + a_{n+1, n} x_n + b_{n+1} x_{n+1} = 0
        \end{aligned}
        \right.
    \]

    这是一个 $n + 1$ 个方程构成的关于 $n + 1$ 个未知数的齐次线性方程组, 所以其一定有解. 而如果原方程组有解, $x_1$ 到 $x_n$ 取原方程组的解, 并取 $x_{n + 1} = -1$, 就成了这个线性方程组的一组非零解. 这说明这个线性方程组至少有两组解: 零解和 $x_{n + 1} = -1$ 的解. 所以这个方程组没有唯一解, 因此其系数矩阵不可逆, 行列式为 $0$.
\end{proof}
不充分的例子: $n = 1$, $a_{11} = a_{21} = 0$, $b_1 = b_2 = 1$. 那么
\[
\left|\begin{matrix}
    0 & 1 \\ 0 & 1
\end{matrix}\right| = 0,
\]
但是方程组是两个 $0 = 1$ 的方程, 没有解.

\newpage
\subsection*{89页 题目8}
\begin{problem*}
    设 $\mat{A}$ 是 $n(n > 2)$ 阶方阵. 证明 $|\mat{A}^{\ast}| = |\mat{A}|^{n - 1}$.
\end{problem*}
\begin{proof}
    如果 $\mat{A}$ 可逆, 那么 $|\mat{A}^\ast| = |\mat{A}|^n|\mat{A}^{-1}| = |\mat{A}|^{n - 1} $; 如果 $\mat{A}$ 不可逆, 那么如果 $|\mat{A}^\ast| \not= 0$, 就存在 $\mat{B}$ 使 $\mat{A}^\ast \mat{B} = \mat{E}$. 而 $\mat{A}\mat{A}^\ast = \mat{O}$, 所以 $\mat{A} = \mat{A}(\mat{A}^\ast\mat{B}) = (\mat{A}\mat{A})^\ast\mat{B} = \mat{O}$. 而此时 $\mat{A}^\ast = \mat{O}$, 与假设 $|\mat{A}^\ast| \not= 0$ 矛盾. 所以 $\mat{A}$ 不可逆时, $|\mat{A}^\ast| = 0 = |\mat{A}|^{n - 1}$. 命题得证.
\end{proof}